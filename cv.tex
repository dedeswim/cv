%!TEX TS-program = xelatex
%!TEX encoding = UTF-8 Unicode
% Awesome CV LaTeX Template for CV/Resume
%
% This template has been downloaded from:
% https://github.com/posquit0/Awesome-CV
%
% Author:
% Claud D. Park <posquit0.bj@gmail.com>
% http://www.posquit0.com
%
% Template license:
% CC BY-SA 4.0 (https://creativecommons.org/licenses/by-sa/4.0/)
%


%-------------------------------------------------------------------------------
% CONFIGURATIONS
%-------------------------------------------------------------------------------
% A4 paper size by default, use 'letterpaper' for US letter
\documentclass[11pt, a4paper]{awesome-cv}

\usepackage{tabularx}
\usepackage{array}
\usepackage{fontawesome}

\usepackage{xstring}
\usepackage{etoolbox}
\newboolean{bold}

\addbibresource{publications.bib}

\renewcommand*{\mkbibnamegiven}[1]{%
  \ifitemannotation{highlight}
    {\textbf{#1}}
    {#1}}

\renewcommand*{\mkbibnamefamily}[1]{%
  \ifitemannotation{highlight}
    {\textbf{#1}}
    {#1}}
    
\usepackage{xeCJK}
\setCJKmainfont[Scale=0.8]{Noto Sans CJK SC}

% Configure page margins with geometry
\geometry{left=1.4cm, top=.8cm, right=1.4cm, bottom=.8cm, footskip=.5cm}

% Specify the location of the included fonts
\fontdir[fonts/]

% Color for highlights
% Awesome Colors: awesome-emerald, awesome-skyblue, awesome-red, awesome-pink, awesome-orange
%                 awesome-nephritis, awesome-concrete, awesome-darknight
% \colorlet{awesome}{awesome-darknight}
% Uncomment if you would like to specify your own color
\definecolor{awesome}{HTML}{0085ff}
\hypersetup{
    colorlinks=true,
    urlcolor=awesome,
    linkbordercolor=black,
    pdfborderstyle={/S/U/W 1}
}


% Colors for text
% Uncomment if you would like to specify your own color
% \definecolor{darktext}{HTML}{414141}
% \definecolor{text}{HTML}{333333}
% \definecolor{graytext}{HTML}{5D5D5D}
% \definecolor{lighttext}{HTML}{999999}

% Set false if you don't want to highlight section with awesome color
\setbool{acvSectionColorHighlight}{false}

% If you would like to change the social information separator from a pipe (|) to something else
\renewcommand{\acvHeaderSocialSep}{\quad\textbar\quad}


%-------------------------------------------------------------------------------
%	PERSONAL INFORMATION
%	Comment any of the lines below if they are not required
%-------------------------------------------------------------------------------
% Available options: circle|rectangle,edge/noedge,left/right
% \photo[rectangle,edge,right]{./examples/profile}
\name{Edoardo}{Debenedetti} 
\position{PhD Student in CS @ ETH Z{\"u}rich}
% \address{42-8, Bangbae-ro 15-gil, Seocho-gu, Seoul, 00681, Rep. of KOREA}

\mobile{(+39) 340 512 6541}
\email{edoardo.m.debenedetti@gmail.com}
\homepage{edoardo.science}
\github{dedeswim}
\linkedin{edoardo-debenedetti}
% \gitlab{gitlab-id}
% \stackoverflow{SO-id}{SO-name}
% \twitter{@twit}
% \skype{skype-id}
% \reddit{reddit-id}
% \medium{madium-id}
\googlescholar{6Urve9wAAAAJ}{}
%% \firstname and \lastname will be used
% \googlescholar{googlescholar-id}{}
% \extrainfo{extra informations}


% \quote{``Be the change that you want to see in the world."}


%-------------------------------------------------------------------------------
\begin{document}

% Print the header with above personal informations
% Give optional argument to change alignment(C: center, L: left, R: right)
\makecvheader[L]

% Print the footer with 3 arguments(<left>, <center>, <right>)
% Leave any of these blank if they are not needed
\makecvfooter
  {}
  {}
  {}


%-------------------------------------------------------------------------------
%	CV/RESUME CONTENT
%	Each section is imported separately, open each file in turn to modify content
%-------------------------------------------------------------------------------
% \input{resume/summary.tex}
%-------------------------------------------------------------------------------
%	SECTION TITLE
%-------------------------------------------------------------------------------
\cvsection{Education}


%-------------------------------------------------------------------------------
%	CONTENT
%-------------------------------------------------------------------------------
\begin{cventries}

  %---------------------------------------------------------
  \cventry
  {MSc in Computer Science} % Degree
  {EPFL - Federal Institute of Technology Lausanne} % Institution
  {Lausanne, Switzerland} % Location
  {Sep. 2019 - Apr. 2022 (Expected)} % Date(s)
  {
    \begin{cvitems} % Description(s) bullet points
      \item \textbf{GPA 5.51/6}, focus on \textbf{Machine Learning $\cap$ Security $\cap$ Privacy}.
      \item Working on my Master's Thesis about the robustness of Vision Transformers at \textbf{Princeton University}, in \textbf{Prof. Mittal}'s lab.
      \item Worked on \textbf{RobustBench}, a standardized benchmark for \textbf{Adversarial Robustness} at \textbf{Prof. Flammarion}'s TML Lab.
      \item Worked on a \textbf{research} project about \textbf{deepfakes} counteraction via influence functions at \textbf{Prof. Troncoso}'s SPRING Lab.
    \end{cvitems}
  }
  \cventry
  {BSc in Computer Engineering} % Degree
  {Politecnico di Torino} % Institution
  {Turin, Italy} % Location
  {Sep. 2016 - Jul. 2019} % Date(s)
  {
    \begin{cvitems} % Description(s) bullet points
      \item \textbf{GPA 28.4/30}, graduation mark 110/110, \textbf{top 9\%}.
      \item \textbf{Exchange year at 同济大学} (Tongji University), in Shanghai (China), supported by a \textbf{full scholarship} granted to the top 31\% applicants. \vspace{0.2cm}
    \end{cvitems}
  }
  \cventry
  {High School Diploma} % Degree
  {Navy Military College ``F. Morosini"} % Institution
  {Venice, Italy} % Location
  {Sep. 2013 - Jul. 2016} % Date(s)
  {
    \begin{cvitems} % Description(s) bullet points
      \item Selected to lead sophomores as \textbf{prefect} during my final year.
      \item \textbf{Military training} on Italian Navy's Ships and at Italian Navy's Marine Corps.
    \end{cvitems}
  }

  %---------------------------------------------------------
\end{cventries}

%-------------------------------------------------------------------------------
%	SECTION TITLE
%-------------------------------------------------------------------------------
\cvsection{Experience}


%-------------------------------------------------------------------------------
%	CONTENT
%-------------------------------------------------------------------------------
\begin{cventries}

  %---------------------------------------------------------
  \cventry
  {Software Engineering Intern} % Job title
  {Bloomberg LP} % Organization
  {London, UK} % Location
  {Jul 2021 - Sep. 2021} % Date(s)
  {
    \begin{cvitems} % Description(s) of tasks/responsibilities
      \item Incoming Software Engineering Intern, project TBD in the next few months.
    \end{cvitems}
  }

  %---------------------------------------------------------
  \cventry
  {Research Intern} % Job title
  {armasuisse Cyber-Defence Campus} % Organization
  {Lausanne, Switzerland} % Location
  {Aug. 2020 - Feb. 2021} % Date(s)
  {
    \begin{cvitems} % Description(s) of tasks/responsibilities
      \item Conducted research about \textbf{Machine Unlearning} and \textbf{Membership Inference Attacks} against Generative Models under the supervision of \textbf{Dr. Mathias Humbert}. Tools: \textbf{PyTorch}, \textbf{PyTorch Lightning}, \textbf{Comet ML}, and \textbf{Python}.
    \end{cvitems}
  }

  %---------------------------------------------------------
  \cventry
  {Software Engineering Intern} % Job title
  {Reply} % Organization
  {Turin, Italy} % Location
  {Nov. 2018 - Feb. 2019} % Date(s)
  {
    \begin{cvitems} % Description(s) of tasks/responsibilities
      \item Developed a \textbf{chatbot} that answers questions about GDPR law, using \textbf{TypeScript}, \textbf{Redis}, \textbf{MongoDB}, \textbf{IBM Watson Assistant}, and \textbf{Docker}.
      \item Worked on \textbf{RPA}, using \textbf{Python}. One of the bots I developed \textbf{decreased a task duration by 88\%,} without requiring human intervention in it.
    \end{cvitems}
  }

\end{cventries}

%-------------------------------------------------------------------------------
%	SECTION TITLE
%-------------------------------------------------------------------------------
\cvsection{Publication}


%-------------------------------------------------------------------------------
%	CONTENT
%-------------------------------------------------------------------------------



\begin{cvparagraph}
\vspace{5.0mm}
\begin{cvitems}
    \item Croce*, F., Andriushchenko*, M., Sehwag*, V. \textbf{Debenedetti*, E.}, Flammarion, N., Chiang, M., Mittal, P., Hein, M., \textit{``RobustBench: a standardized adversarial robustness benchmark''}, Thirty‑fifth Conference on Neural Information Processing Systems Datasets and Benchmarks Track, 2021, URL: \url{https://openreview.net/forum?id=SSKZPJCt7B}. \textbf{50+ citations}. \textit{(* equal contribution)}
\end{cvitems}
\end{cvparagraph}
\vspace{2.5mm}

%-------------------------------------------------------------------------------
%	SECTION TITLE
%-------------------------------------------------------------------------------
\cvsection{Honors}

%-------------------------------------------------------------------------------
%	CONTENT
%-------------------------------------------------------------------------------
\begin{cvhonors}

%---------------------------------------------------------
  \cvhonor
    {Best Paper Honorable Mention Prize} % Award
    {ICLR 2021 Workshop on Security and Safety in Machine Learning Systems, for a preliminary version of the RobustBench paper.} % Event
    {Virtual} % Location
    {2021} % Date(s)

%---------------------------------------------------------
\end{cvhonors}

%-------------------------------------------------------------------------------
%	SECTION TITLE
%-------------------------------------------------------------------------------
\cvsection{Projects}


%-------------------------------------------------------------------------------
%	CONTENT
%-------------------------------------------------------------------------------
\begin{cventries}
  %---------------------------------------------------------
  \cventry
  {Counteracting DeepFakes} % Job title
  {} % Organization
  {} % Location
  {Graded 5.75/6} % Date(s)
  {
    \begin{cvitems} % Description(s) of tasks/responsibilities
      \item Term \textbf{Research Project} run at Prof. Troncoso's SPRING Lab at EPFL.
      \item Attempt to run \textbf{poisoning} attacks via \textbf{Influence Functions} (by Koh et al.) to counteract DeepFakes training.
      \item The attack was tested on MNIST-trained \textbf{autoencoder} and shown \textbf{major distortions} in the decoder outputs after the poisoned training.
      \item Stack: \textbf{PyTorch}, \textbf{TensorFlow} and \textbf{Keras}, on top of \textbf{Python}.
    \end{cvitems}
  }


  % ---------------------------------------------------------
  % \cventry
  % {Database Systems Project} % Job title
  % {} % Organization
  % {} % Location
  % {Graded 6/6} % Date(s)
  % {
  %   \begin{cvitems} % Description(s) of tasks/responsibilities
  %     \item Implementation of \textbf{Jaccard similarity based LSH for approximate K-NNs}.
  %     \item Implementation of a \textbf{MapReduce-friendly theta-join} according to the 1-Bucket-Theta algorithm with additional Reduce optimizations.
  %     \item Stack: \textbf{Scala} and \textbf{Spark}.
  %   \end{cvitems}
  % }

  %---------------------------------------------------------
  \cventry
  {NeurIPS 2019 Reproducibility Report} % Job title
  {} % Organization
  {} % Location
  {Graded 6/6} % Date(s)
  {
    \begin{cvitems} % Description(s) of tasks/responsibilities
      \item Reproducibility report of a NeurIPS 2019 paper about \textbf{black-box adversarial attacks against ML models}, submitted to the NeurIPS 2019 Reproducibility Challenge.
      \item With our re-implementation we \textbf{reduced by 76\% the median attack time} wrt the paper. Stack: \textbf{Python}, \textbf{PyTorch}, \LaTeX.
    \end{cvitems}
  }

\end{cventries}

% \input{resume/presentation.tex}
% \input{resume/writing.tex}
% \input{resume/committees.tex}
%-------------------------------------------------------------------------------
%	SECTION TITLE
%-------------------------------------------------------------------------------
\cvsection{Extra}


%-------------------------------------------------------------------------------
%	CONTENT
%-------------------------------------------------------------------------------
\begin{cventries}

%---------------------------------------------------------
  \cventry
    {International Manager} % Affiliation/role
    {JEToP (PoliTo's Junior Enterprise)} % Organization/group
    {Turin, Italy} % Location
    {Oct. 2018 - Jun. 2019} % Date(s)
    {
      \begin{cvitems} % Description(s) of experience/contributions/knowledge
        \item {\textbf{Executive board} member responsible for external relations of the Junior Enterprise. Signed \textbf{8 new partnerships.}}
        \item {Past positions (since Oct. 2016): Fundraising \& Partnership Assistant, IT Consultant.}
      \end{cvitems}
    }
  \cventry
    {Mentee} % Affiliation/role
    {LeadTheFuture} % Organization/group
    {} % Location
    {Sept. 2019 - Current} % Date(s)
    {
      \begin{cvitems} % Description(s) of experience/contributions/knowledge
        \item {Selected to be part of the \textbf{leading mentorship organization for STEM students in Italy} - Acceptance rate < 20\%.}
        \item {Held a \textbf{mentoring session} about MSc admissions at EPFL.}
      \end{cvitems}
    }
%---------------------------------------------------------
\end{cventries}

% %-------------------------------------------------------------------------------
%	SECTION TITLE
%-------------------------------------------------------------------------------
\cvsection{Languages}

%-------------------------------------------------------------------------------
%	CONTENT
%-------------------------------------------------------------------------------
\begin{cvparagraph}
\begin{tabularx}{\textwidth}{ 
  >{\raggedright\arraybackslash}X 
  >{\centering\arraybackslash}X 
  >{\raggedleft\arraybackslash}X }
     \textbullet\ \textbf{English}: Fluent, TOEFL iBT 108/120, C1. & \textbullet\ \textbf{Italian}: Native speaker. & \textbullet\ \textbf{French}: Intermediate.
\end{tabularx}
\end{cvparagraph}


%-------------------------------------------------------------------------------
\end{document}
