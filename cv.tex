%!TEX TS-program = xelatex
%!TEX encoding = UTF-8 Unicode
% Awesome CV LaTeX Template for CV/Resume
%
% This template has been downloaded from:
% https://github.com/posquit0/Awesome-CV
%
% Author:
% Claud D. Park <posquit0.bj@gmail.com>
% http://www.posquit0.com
%
% Template license:
% CC BY-SA 4.0 (https://creativecommons.org/licenses/by-sa/4.0/)
%


%-------------------------------------------------------------------------------
% CONFIGURATIONS
%-------------------------------------------------------------------------------
% A4 paper size by default, use 'letterpaper' for US letter
\documentclass[11pt, a4paper]{awesome-cv}

\usepackage{tabularx}
\usepackage{array}
\usepackage{fontawesome}

\usepackage{xstring}
\usepackage{etoolbox}
\newboolean{bold}

\addbibresource{publications.bib}

\renewcommand*{\mkbibnamegiven}[1]{%
  \ifitemannotation{highlight}
    {\textbf{#1}}
    {#1}}

\renewcommand*{\mkbibnamefamily}[1]{%
  \ifitemannotation{highlight}
    {\textbf{#1}}
    {#1}}
    
\usepackage{xeCJK}
\setCJKmainfont[Scale=0.8]{Noto Sans CJK SC}

% Configure page margins with geometry
\geometry{left=1.4cm, top=.8cm, right=1.4cm, bottom=.8cm, footskip=.5cm}

% Specify the location of the included fonts
\fontdir[fonts/]

% Color for highlights
% Awesome Colors: awesome-emerald, awesome-skyblue, awesome-red, awesome-pink, awesome-orange
%                 awesome-nephritis, awesome-concrete, awesome-darknight
% \colorlet{awesome}{awesome-darknight}
% Uncomment if you would like to specify your own color
\definecolor{awesome}{HTML}{0085ff}
\hypersetup{
    colorlinks=true,
    urlcolor=awesome,
    linkbordercolor=black,
    pdfborderstyle={/S/U/W 1}
}


% Colors for text
% Uncomment if you would like to specify your own color
% \definecolor{darktext}{HTML}{414141}
% \definecolor{text}{HTML}{333333}
% \definecolor{graytext}{HTML}{5D5D5D}
% \definecolor{lighttext}{HTML}{999999}

% Set false if you don't want to highlight section with awesome color
\setbool{acvSectionColorHighlight}{false}

% If you would like to change the social information separator from a pipe (|) to something else
\renewcommand{\acvHeaderSocialSep}{\quad\textbar\quad}


%-------------------------------------------------------------------------------
%	PERSONAL INFORMATION
%	Comment any of the lines below if they are not required
%-------------------------------------------------------------------------------
% Available options: circle|rectangle,edge/noedge,left/right
% \photo[rectangle,edge,right]{./examples/profile}
\name{Edoardo}{Debenedetti} 
\position{PhD Student in CS @ ETH Z{\"u}rich}
% \address{42-8, Bangbae-ro 15-gil, Seocho-gu, Seoul, 00681, Rep. of KOREA}

\mobile{(+39) 340 512 6541}
\email{edoardo.m.debenedetti@gmail.com}
\homepage{edoardo.science}
\github{dedeswim}
\linkedin{edoardo-debenedetti}
% \gitlab{gitlab-id}
% \stackoverflow{SO-id}{SO-name}
% \twitter{@twit}
% \skype{skype-id}
% \reddit{reddit-id}
% \medium{madium-id}
\googlescholar{6Urve9wAAAAJ}{}
%% \firstname and \lastname will be used
% \googlescholar{googlescholar-id}{}
% \extrainfo{extra informations}


% \quote{``Be the change that you want to see in the world."}


%-------------------------------------------------------------------------------
\begin{document}

% Print the header with above personal informations
% Give optional argument to change alignment(C: center, L: left, R: right)
\makecvheader[L]

% Print the footer with 3 arguments(<left>, <center>, <right>)
% Leave any of these blank if they are not needed
\makecvfooter
  {}
  {}
  {}


%-------------------------------------------------------------------------------
%	CV/RESUME CONTENT
%	Each section is imported separately, open each file in turn to modify content
%-------------------------------------------------------------------------------
% \begin{cvparagraph}

%---------------------------------------------------------
Lorem ipsum dolor sit amet, consectetur adipiscing elit, sed do eiusmod tempor incididunt ut labore et dolore magna aliqua. Ut enim ad minim. Lorem ipsum dolor sit amet, consectetur adipiscing elit, sed do eiusmod tempor incididunt ut labore et dolore magna aliqua. Ut enim ad minim.
\end{cvparagraph}

%-------------------------------------------------------------------------------
%	SECTION TITLE
%-------------------------------------------------------------------------------
\cvsection{Education}


%-------------------------------------------------------------------------------
%	CONTENT
%-------------------------------------------------------------------------------
\begin{cventries}

  %---------------------------------------------------------
  \cventry
  {MSc in Computer Science} % Degree
  {EPFL - Federal Institute of Technology Lausanne} % Institution
  {Lausanne, Switzerland} % Location
  {Sep. 2019 - Apr. 2022 (Expected)} % Date(s)
  {
    \begin{cvitems} % Description(s) bullet points
      \item \textbf{GPA 5.51/6}, focus on \textbf{Machine Learning $\cap$ Security $\cap$ Privacy}.
      \item Working on my Master's Thesis about the robustness of Vision Transformers at \textbf{Princeton University}, in \textbf{Prof. Mittal}'s lab.
      \item Worked on \textbf{RobustBench}, a standardized benchmark for \textbf{Adversarial Robustness} at \textbf{Prof. Flammarion}'s TML Lab.
      \item Worked on a \textbf{research} project about \textbf{deepfakes} counteraction via influence functions at \textbf{Prof. Troncoso}'s SPRING Lab.
    \end{cvitems}
  }
  \cventry
  {BSc in Computer Engineering} % Degree
  {Politecnico di Torino} % Institution
  {Turin, Italy} % Location
  {Sep. 2016 - Jul. 2019} % Date(s)
  {
    \begin{cvitems} % Description(s) bullet points
      \item \textbf{GPA 28.4/30}, graduation mark 110/110, \textbf{top 9\%}.
      \item \textbf{Exchange year at 同济大学} (Tongji University), in Shanghai (China), supported by a \textbf{full scholarship} granted to the top 31\% applicants. \vspace{0.2cm}
    \end{cvitems}
  }
  \cventry
  {High School Diploma} % Degree
  {Navy Military College ``F. Morosini"} % Institution
  {Venice, Italy} % Location
  {Sep. 2013 - Jul. 2016} % Date(s)
  {
    \begin{cvitems} % Description(s) bullet points
      \item Selected to lead sophomores as \textbf{prefect} during my final year.
      \item \textbf{Military training} on Italian Navy's Ships and at Italian Navy's Marine Corps.
    \end{cvitems}
  }

  %---------------------------------------------------------
\end{cventries}

%-------------------------------------------------------------------------------
%	SECTION TITLE
%-------------------------------------------------------------------------------
\cvsection{Experience}


%-------------------------------------------------------------------------------
%	CONTENT
%-------------------------------------------------------------------------------
\begin{cventries}

  %---------------------------------------------------------
  \cventry
  {Software Engineering Intern} % Job title
  {Bloomberg LP} % Organization
  {London, UK} % Location
  {Jul. 2021 - Sep. 2021} % Date(s)
  {
    \begin{cvitems} % Description(s) of tasks/responsibilities
      \item Worked in the \textbf{Multi Asset Risk System} team, on the re-design and implementation of the configuration of a distributed logging library.
      \item Move the configuration of a \textbf{distributed logging library} from an internal technology to a \textbf{centralized SQL DB}, using a \textbf{cache} and a \textbf{C++ service}.
      % \item Used  to fetch the configuration and serve it to the client library.
      \item The configuration is checked \textbf{\textasciitilde 1M times per minute}, and the usage of the cache gave a \textbf{\textasciitilde 23x speed improvement} w.r.t. querying the DB.
    \end{cvitems}
  }

  %---------------------------------------------------------
  \cventry
  {Research Intern} % Job title
  {armasuisse Cyber-Defence Campus} % Organization
  {Lausanne, Switzerland} % Location
  {Aug. 2020 - Feb. 2021} % Date(s)
  {
    \begin{cvitems} % Description(s) of tasks/responsibilities
      \item Worked on \textbf{Machine Unlearning} and \textbf{Membership Inference Attacks} against Generative Models, supervised by \textbf{Prof. Mathias Humbert}.
      \item Adapt the \textbf{MIA} technique proposed by the \textit{GAN-Leaks} work (by Chen at al.), to work after the removal some datapoints from the training set.
      \item The technique achieved \textbf{promising results} when attacking DCGAN trained on the CelebA dataset
    \end{cvitems}
  }

  %---------------------------------------------------------
  \cventry
  {Software Engineering Intern} % Job title
  {Reply} % Organization
  {Turin, Italy} % Location
  {Nov. 2018 - Feb. 2019} % Date(s)
  {
    \begin{cvitems} % Description(s) of tasks/responsibilities
      \item Developed a \textbf{chatbot} that answers questions about GDPR law, using \textbf{TypeScript}, \textbf{Redis}, \textbf{MongoDB}, \textbf{IBM Watson Assistant}, and \textbf{Docker}.
      \item Worked on \textbf{RPA}, using \textbf{Python}. One of the bots \textbf{decreased a task duration by 88\%,} without requiring human intervention in it.
    \end{cvitems}
  }

\end{cventries}

%-------------------------------------------------------------------------------
%	SECTION TITLE
%-------------------------------------------------------------------------------
\cvsection{Publication}


%-------------------------------------------------------------------------------
%	CONTENT
%-------------------------------------------------------------------------------



\begin{cvparagraph}
\vspace{5.0mm}
\begin{cvitems}
    \item Croce*, F., Andriushchenko*, M., Sehwag*, V. \textbf{Debenedetti*, E.}, Flammarion, N., Chiang, M., Mittal, P., Hein, M., \textit{``RobustBench: a standardized adversarial robustness benchmark''}, Thirty‑fifth Conference on Neural Information Processing Systems Datasets and Benchmarks Track, 2021, URL: \url{https://openreview.net/forum?id=SSKZPJCt7B}. \textbf{50+ citations}. \textit{(* equal contribution)}
\end{cvitems}
\end{cvparagraph}
\vspace{2.5mm}

%-------------------------------------------------------------------------------
%	SECTION TITLE
%-------------------------------------------------------------------------------
\cvsection{Honors}

%-------------------------------------------------------------------------------
%	CONTENT
%-------------------------------------------------------------------------------
\begin{cvhonors}

%---------------------------------------------------------
  \cvhonor
    {PoliTong Full Scholarship} % Award
    {to support the one-year exchange - Top 31\% of the applicants.} % Event
    {PoliTo and TongjiU} % Location
    {2018} % Date(s)
    
  \cvhonor
    {Mentee} % Award
    {selected to be part of the leading mentorship organization for STEM students in Italy - Acceptance rate < 20\%.} % Event
    {LeadTheFuture} % Location
    {2019} % Date(s)

%---------------------------------------------------------
\end{cvhonors}

%-------------------------------------------------------------------------------
%	SECTION TITLE
%-------------------------------------------------------------------------------
\cvsection{Projects}


%-------------------------------------------------------------------------------
%	CONTENT
%-------------------------------------------------------------------------------
\begin{cventries}
  %---------------------------------------------------------
  \cventry
  {RobustBench: a standardized adversarial robustness benchmark} % Job title
  {} % Organization
  {} % Location
  {EPFL, Graded 6/6} % Date(s)
  {
    \begin{cvitems} % Description(s) of tasks/responsibilities
      \item \textbf{Benchmark and leaderboard} to analyze the \textbf{robustness} of image classifiers, and \textbf{Model Zoo} containing 60+ \textbf{PyTorch} models trained robustly.
      \item \textbf{Analysis} on the progress in robustness: worked on the \textbf{Lipschitzness} of robust models, and on a new \textbf{ensemble black-box transfer attack}, capable of \textbf{improving the success rate by up to 70\%} w.r.t. transfer attacks carried with one model only.
    \end{cvitems}
  }
  
  %---------------------------------------------------------
  \cventry
  {Counteracting DeepFakes} % Job title
  {} % Organization
  {} % Location
  {EPFL, Graded 5.75/6} % Date(s)
  {
    \begin{cvitems} % Description(s) of tasks/responsibilities
      \item Term \textbf{Research Project} run at \textbf{Prof. Troncoso}'s SPRING Lab at EPFL, worked on \textbf{poisoning} attacks via \textbf{Influence Functions} against DeepFakes.
      \item Tested the attack on MNIST-trained \textbf{autoencoder} which showed \textbf{major distortions} in its output after the poisoned training.
      % \item Stack: \textbf{PyTorch}, \textbf{TensorFlow} and \textbf{Keras}, on top of \textbf{Python}.
    \end{cvitems}
  }
  
  % \cventry
  % {Membership Inference Attacks against ``unlearned" GANs} % Job title
  % {} % Organization
  % {} % Location
  % {armasuisse CYD Campus} % Date(s)
  % {
  %   \begin{cvitems} % Description(s) of tasks/responsibilities
  %     \item Adapt the \textbf{MIA} technique proposed by the \textit{GAN-Leaks} work (by Chen at al.), to work after the removal some % datapoints from the training set.
  %     % \item Provide intuition of why the \textbf{unlearning setting makes membership inference easier}.
  %     \item The technique achieved \textbf{promising results} when attacking DCGAN trained on the CelebA dataset
  %     % \item Stack: \textbf{PyTorch}, \textbf{PyTorch Lightning}, \textbf{CometML}, on top of \textbf{Python}.
  %   \end{cvitems}
  % }

  %\cventry
  %{Distributed logging library configuration} % Job title
  %{} % Organization
  %{} % Location
  %{Bloomberg LP} % Date(s)
  %{
  %  \begin{cvitems} % Description(s) of tasks/responsibilities
  %    \item Move the configuration of a \textbf{distributed logging library} from an internal technology to a \textbf{centralized SQL DB}, %using a \textbf{cache} and a \textbf{C++ service}.
  %    % \item Used  to fetch the configuration and serve it to the client library.
  %    \item The configuration is checked \textbf{\textasciitilde 1M times per minute}, and the usage of the cache gave a %\textbf{\textasciitilde 23x speed improvement} w.r.t. querying the DB.
  %    % \item Stack: \textbf{C++03}, \textbf{C++17}, \textbf{Comdb2} (SQL).
  %  \end{cvitems}
  %}

  % ---------------------------------------------------------
  % \cventry
  % {Database Systems Project} % Job title
  % {} % Organization
  % {} % Location
  % {Graded 6/6} % Date(s)
  % {
  %   \begin{cvitems} % Description(s) of tasks/responsibilities
  %     \item Implementation of \textbf{Jaccard similarity based LSH for approximate K-NNs}.
  %     \item Implementation of a \textbf{MapReduce-friendly theta-join} according to the 1-Bucket-Theta algorithm with additional Reduce % optimizations.
  %     \item Stack: \textbf{Scala} and \textbf{Spark}.
  %   \end{cvitems}
  % }
  
    % ---------------------------------------------------------

  %---------------------------------------------------------
  % \cventry
  % {NeurIPS 2019 Reproducibility Report} % Job title
  % {} % Organization
  % {} % Location
  % {Graded 6/6} % Date(s)
  % {
  %   \begin{cvitems} % Description(s) of tasks/responsibilities
  %     \item Reproducibility report of a NeurIPS 2019 paper about \textbf{black-box adversarial attacks against ML models}, submitted to % the NeurIPS 2019 Reproducibility Challenge.
  %     \item With our re-implementation we \textbf{reduced by 76\% the median attack time} wrt the paper. Stack: \textbf{Python}, % \textbf{PyTorch}, \LaTeX.
  %   \end{cvitems}
  % }

\end{cventries}

% %-------------------------------------------------------------------------------
%	SECTION TITLE
%-------------------------------------------------------------------------------
\cvsection{Presentation}


%-------------------------------------------------------------------------------
%	CONTENT
%-------------------------------------------------------------------------------
\begin{cventries}

%---------------------------------------------------------
  \cventry
    {Presenter for <Hosting Web Application for Free utilizing GitHub, Netlify and CloudFlare>} % Role
    {DevFest Seoul by Google Developer Group Korea} % Event
    {Seoul, S.Korea} % Location
    {Nov. 2017} % Date(s)
    {
      \begin{cvitems} % Description(s)
        \item {Introduced the history of web technology and the JAM stack which is for the modern web application development.}
        \item {Introduced how to freely host the web application with high performance utilizing global CDN services.}
      \end{cvitems}
    }

%---------------------------------------------------------
  \cventry
    {Presenter for <DEFCON 20th : The way to go to Las Vegas>} % Role
    {6th CodeEngn (Reverse Engineering Conference)} % Event
    {Seoul, S.Korea} % Location
    {Jul. 2012} % Date(s)
    {
      \begin{cvitems} % Description(s)
        \item {Introduced CTF(Capture the Flag) hacking competition and advanced techniques and strategy for CTF}
      \end{cvitems}
    }

%---------------------------------------------------------
\end{cventries}

% %-------------------------------------------------------------------------------
%	SECTION TITLE
%-------------------------------------------------------------------------------
\cvsection{Writing}


%-------------------------------------------------------------------------------
%	CONTENT
%-------------------------------------------------------------------------------
\begin{cventries}

%---------------------------------------------------------
  \cventry
    {Founder \& Writer} % Role
    {A Guide for Developers in Start-up} % Title
    {Facebook Page} % Location
    {Jan. 2015 - PRESENT} % Date(s)
    {
      \begin{cvitems} % Description(s)
        \item {Drafted daily news for developers in Korea about IT technologies, issues about start-up.}
      \end{cvitems}
    }

%---------------------------------------------------------
  \cventry
    {Undergraduate Student Reporter} % Role
    {AhnLab} % Title
    {S.Korea} % Location
    {Oct. 2012 - Jul. 2013} % Date(s)
    {
      \begin{cvitems} % Description(s)
        \item {Drafted reports about IT trends and Security issues on AhnLab Company magazine.}
      \end{cvitems}
    }

%---------------------------------------------------------
\end{cventries}

% %-------------------------------------------------------------------------------
%	SECTION TITLE
%-------------------------------------------------------------------------------
\cvsection{Program Committees}


%-------------------------------------------------------------------------------
%	CONTENT
%-------------------------------------------------------------------------------
\begin{cvhonors}

%---------------------------------------------------------
  \cvhonor
    {Problem Writer} % Position
    {2016 CODEGATE Hacking Competition World Final} % Committee
    {S.Korea} % Location
    {2016} % Date(s)

%---------------------------------------------------------
  \cvhonor
    {Organizer \& Co-director} % Position
    {1st POSTECH Hackathon} % Committee
    {S.Korea} % Location
    {2013} % Date(s)

%---------------------------------------------------------
\end{cvhonors}

%-------------------------------------------------------------------------------
%	SECTION TITLE
%-------------------------------------------------------------------------------
\cvsection{Extra}


%-------------------------------------------------------------------------------
%	CONTENT
%-------------------------------------------------------------------------------
\begin{cventries}

%---------------------------------------------------------
  \cventry
    {International Manager} % Affiliation/role
    {JEToP (PoliTo's Junior Enterprise)} % Organization/group
    {Turin, Italy} % Location
    {Oct. 2018 - Jun. 2019} % Date(s)
    {
      \begin{cvitems} % Description(s) of experience/contributions/knowledge
        \item {\textbf{Executive board} member responsible for external relations of the Junior Enterprise. Signed \textbf{8 new partnerships.}}
        \item {Past positions (since Oct. 2016): Fundraising \& Partnership Assistant, IT Consultant.}
      \end{cvitems}
    }
  \cventry
    {Mentee} % Affiliation/role
    {LeadTheFuture} % Organization/group
    {} % Location
    {Sept. 2019 - Current} % Date(s)
    {
      \begin{cvitems} % Description(s) of experience/contributions/knowledge
        \item {Selected to be part of the \textbf{leading mentorship organization for STEM students in Italy} - Acceptance rate < 20\%.}
        \item {Held a \textbf{mentoring session} about MSc admissions at EPFL.}
      \end{cvitems}
    }
%---------------------------------------------------------
\end{cventries}

% %-------------------------------------------------------------------------------
%	SECTION TITLE
%-------------------------------------------------------------------------------
\cvsection{Languages}

%-------------------------------------------------------------------------------
%	CONTENT
%-------------------------------------------------------------------------------
\begin{cvparagraph}
\begin{tabularx}{\textwidth}{ 
  >{\raggedright\arraybackslash}X 
  >{\centering\arraybackslash}X 
  >{\raggedleft\arraybackslash}X }
     \textbullet\ \textbf{English}: Fluent, TOEFL iBT 108/120, C1. & \textbullet\ \textbf{Italian}: Native speaker. & \textbullet\ \textbf{French}: Intermediate.
\end{tabularx}
\end{cvparagraph}


%-------------------------------------------------------------------------------
\end{document}
